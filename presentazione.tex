%% Direttive TeXworks:
% !TeX root = ./presentazione.tex
% !TEX encoding = UTF-8 Unicode
% !TEX program = arara
% !TEX TS-program = arara
% !TeX spellcheck = it-IT

% arara: pdflatex: { synctex: yes, action: nonstopmode, options: "-halt-on-error -file-line-error-style" }

\documentclass[%
    % handout,      % serve per generare la versione stampabile
    % italian
]{beamer}

%% ORDINE IMPORTANTE INIZIO %%%%%%%%%%%%
\usepackage[T1]{fontenc}        % serve per impostare la codifica di output del font
\usepackage{textcomp}           % serve per fornire supporto ai Text Companion fonts
\usepackage[utf8]{inputenc}     % serve per impostare la codifica di input del font
\usepackage[
    english,        % utilizza l'inglese come lingua secondaria
    italian         % utlizza l'italiano come lingua primaria
]{babel}                        % serve per scrivere Indice, Capitolo, etc in Italiano
\usepackage{lmodern}            % carica una variante Latin Modern prodotto dal GUST
%% ORDINE IMPORTANTE FINE %%%%%%%%%%%%%%
\usepackage{xcolor}             % serve per la gestione dei colori nel testo
\usepackage{graphicx}           % serve per includere immagini e grafici

\usepackage[%
    strict,             % rende tutti gli warning degli errori
    autostyle,          % imposta lo stile in base al linguaggio specificato in babel
    english=american,   % imposta lo stile per l'inglese
    italian=guillemets  % imposta lo stile per l'italiano
]{csquotes}                     % serve a impostare lo stile delle virgolette

% \usetheme{CambridgeUS}        % TODO decidi quale usare
\usetheme{Boadilla}             % serve per scegliere il layout generale dei frame
\usecolortheme{beaver}          % Per il colore va comunque bene questo

\usepackage{pgf}
% \usepackage{pgfpages}         % serve per mettere più slide in una pagina
% \pgfpagesuselayout{4 on 1}[a4paper, border shrink=5mm, landscape]

% \usepackage[htt]{hyphenat}      % Enable hyphenation of TT text
\hyphenation{                   % Permette di sillabare bene le parole
    JavaFX
    Swing
    Micro-systems
    Script
    script
    Pack-age
    pack-age
    Manage-ment
    manage-ment
    Com-munity
    com-munity
}

\setcounter{secnumdepth}{2}     % Numera fino alla sottosezione nel corpo del testo
\setcounter{tocdepth}{3}        % Numera fino alla sotto-sottosezione nell'indice

\graphicspath{{img/}}

%% Definisco un nuovo comando per enfatizzare il testo in inglese %%%%%%%%%%%
\newcommand{\engEmph}[1] {\emph{\foreignlanguage{english}#1}}

% hyperref va caricato il più tardi possibile
\usepackage{hyperxmp}           % Simile ad hyperref
\usepackage{hyperref}           % Gestisce tutte le cose ipertestuali del pdf
\hypersetup{%
    pdfpagemode={UseOutlines},
    hidelinks,          % nasconde i collegamenti (non vengono quadrettati)
    hypertexnames=false,
    linktoc=all,        % inserisce i link nell'indice
    bookmarks=true,
    bookmarksopen,
    bookmarksopenlevel=3,
    plainpages=false,
    breaklinks,
    pdfstartview={FitH},
    pdfauthor={Niccolò Maltoni},
    pdfcopyright={Copyright (C) 2017, Niccolò Maltoni},
    pdftitle={Progettazione object-oriented di un'interfaccia grafica JavaFX per il simulatore Alchemist},
    pdfcontactcountry={Italy},
    pdfcontactemail={niccolo.maltoni@studio.unibo.it},
    pdflang={it},
    pdfsubject={Lo scopo di questa tesi è la progettazione e la successiva implementazione di un'interfaccia grafica 2D per il simulatore Alchemist. La nuova interfaccia permette di interagire con la simulazione a tempo di esecuzione e di vedere chiaramente rappresentate informazioni su di essa. In particolare, è supportata una struttura modulare di effetti che rende facilmente osservabili determinate entità del sistema ed eventuali loro proprietà: rispetto alla classe di effetti dell'interfaccia classica, la nuova tipologia di effetti non è più una funzione dal singolo nodo alla rappresentazione grafica, bensì fa riferimento all'intero ambiente, permettendo di costruire rendering complessi facilmente esportabili tramite file JSON. Si è scelto di mantenere un'interfaccia il più possibile user-friendly, mantenendo un design più simile ai simulatori a scopo videoludico per favorire l'utilizzo da parte di utenti inesperti. Lo stile estetico al quale si è deciso di allinearsi è il Material Design di Google e la libreria grafica utilizzata per l'implementazione è stata JavaFX.},
    pdfkeywords={Progettazione object-oriented, Simulazione, Java, JavaFX, Interfaccia grafica},
    pdfa
}

\title[Nuova GUI Alchemist in JavaFX]{%
    Progettazione object-oriented di un'interfaccia grafica %
    JavaFX per il simulatore Alchemist
}
\subtitle{Tesi in Programmazione ad Oggetti}
\author[Niccolò~Maltoni]{Niccolò~Maltoni\\0000719734}
% \author[%
%     Niccolò~Maltoni \hspace{0.1\textwidth} \insertframenumber\,/\,\inserttotalframenumber%
% ]{%
%     Niccolò~Maltoni%
%     \\ \vspace{10pt} \small{Realtore: Prof.~Mirko~Viroli \\Correlatore: Prof.~Danilo~Pianini}%
% }
\date{12 dicembre 2017}
\institute[]{%
    Alma Mater Studiorum - Università di Bologna\\%
    Campus di Cesena%
}

% Permette di inserire l'outline prima di ogni sezione
\AtBeginSection[]{%
    \begin{frame}<beamer>
        \frametitle{Outline}
        \tableofcontents[currentsection]
    \end{frame}
}

\begin{document}
    \begin{frame}
        \titlepage
    \end{frame}

    \section{Introduzione}\label{sec:intro}

    \subsection{Obiettivo}\label{subsec:obiettivo}
    \begin{frame}
        \frametitle{\insertsection}
        \framesubtitle{\insertsubsection}

        Lo scopo di questa tesi è la progettazione e la successiva implementazione di un'interfaccia grafica 2D per il simulatore \emph{Alchemist}.

        \medskip
        \pause

        La nuova interfaccia permette di interagire con la simulazione a tempo di esecuzione e di vedere chiaramente rappresentate informazioni su di essa attraverso un'interfaccia il più possibile \engEmph{user-friendly}.
    \end{frame}

    \subsection{Alchemist}\label{subsec:alchemist}
    \begin{frame}
        \frametitle{\insertsection}
        \framesubtitle{\insertsubsection}

        Alchemist\only<1->{\footnote{\url{http://alchemistsimulator.github.io}}} è un meta-simulatore estendibile completamente open-source che esegue su Java Virtual Machine (JVM), nato all’interno dell’Università di Bologna e reperibile su GitHub\only<1->{\footnote{\url{https://github.com/AlchemistSimulator/Alchemist}}}.
    \end{frame}

    \subsection{Il modello di Alchemist}\label{subsec:alchModel}
    \begin{frame}
        \frametitle{\insertsection}
        \framesubtitle{\insertsubsection}

        \centering
        \includegraphics[scale=.34]{fig/alchemist_model}
    \end{frame}

    \begin{frame}
        \frametitle{\insertsection}
        \framesubtitle{\insertsubsection}

    \end{frame}

    \section{L'interfaccia classica}\label{sec:old}
    \begin{frame}
        \frametitle{\insertsection}
        L'architettura di Alchemist è progettata con paradigma \engEmph{Model-View-Controller} (MVC), di conseguenza la suddivisione tra componente grafica (\engEmph{View}) e il blocco ``logico'' composto da \engEmph{Model} e \engEmph{Controller} è netta.

        \medskip
        \pause

        Questa distinzione è evidente anche per quanto riguarda l'utilizzo pratico del software:

        \begin{itemize}[<+(1)->]
          \item
              una simulazione su Alchemist può venire lanciata da terminale, senza che alcuna interfaccia grafica sia necessaria per tutta la durata del periodo di esecuzione \ldots

          \item
              \ldots oppure essere inizializzata, lanciata e controllata in tempo reale dalla sua interfaccia grafica.
        \end{itemize}
    \end{frame}

    \begin{frame}
        \frametitle{\insertsection}
        \centering
        \frame{\includegraphics[scale=.29]{old/window_stopped_circle}}
    \end{frame}

    \begin{frame}
        \frametitle{\insertsection}

        L'interfaccia utente classica di Alchemist è caratterizzata da un'usabilità appena sufficiente, funzionale alle necessità di un utilizzatore esperto, ma non adeguata a fornire un'esperienza completa e \engEmph{user-friendly} ad un utente ``standard''.
        In particolare:

        \begin{itemize}[<+->]
          \item
              Il sistema di controllo non è intuitivo.
          \item
              L'aspetto estetico è:
              \begin{itemize}[<+(1)->]
                  \item datato;
                  \item non aderisce ad alcun design grafico in particolare;
                  \item definisce un proprio design originale.
              \end{itemize}
          \item
              Le capacità di rappresentazione, rappresentate dagli effetti, sono legate strettamente ai nodi e limitano la libertà di rappresentazione.
        \end{itemize}
    \end{frame}

    \section{L'interfaccia implementata}\label{sec:new}

    \begin{frame}
        \frametitle{\insertsection}
        \centering
        \includegraphics[scale=0.27]{img/new/window_all}
    \end{frame}

    \subsection{Caratteristiche}\label{subsec:feature}
    \begin{frame}
        \frametitle{\insertsection}
        \framesubtitle{\insertsubsection}
        \begin{itemize}[<+(1)->]
            \item
                La libreria utilizzata per la nuova interfaccia è JavaFX.
            \item
                L'ambiente di simulazione è stato rinnovato e permette di avere rappresentazione:
                \begin{itemize}[<+(1)->]
                    \item delle proprietà di tutto l'ambiente attraverso gli \emph{effetti};
                    \item dell'avanzamento della simulazione in termini di tempo e step;
                    \item dello stato corrente della simulazione.
                \end{itemize}

            % \item
            %     La gestione degli effetti risulta immediata all'utilizzo, con bottoni dedicati al salvataggio e al caricamento dei file JSON che aprono l'interfaccia fornita dal \engEmph{file manager} del sistema operativo per la scelta del file.
            %
            %     È possibile modificare l'ordine di effetti e gruppi di effetti nella pila con un semplice \engEmph{drag'n'drop}.
        \end{itemize}
    \end{frame}

    \begin{frame}
        \frametitle{\insertsection}
        \framesubtitle{\insertsubsection}
        \begin{itemize}[<+(1)->]
            \item
                Sono stati realizzati effetti standard di base per la rappresentazione di nodi e collegamenti.

            \item
                L'interfaccia non utilizza valori di misura assoluti e scala senza problemi su diversi tipi di risoluzione e densità di pixel.

            \item
                È possibile controllarne il flusso di esecuzione sia tramite la pressione del tasto ``play/pausa'' che attraverso una scorciatoia da tastiera.

            \item
                È possibile effettuare uno zoom nell'ambiente rappresentato eseguendo uno scroll.
        \end{itemize}
    \end{frame}

    \subsection{Prestazioni}\label{subsec:benchmark}
    \begin{frame}
        \frametitle{\insertsection}
        \framesubtitle{\insertsubsection}
        %% Direttive TeXworks:
% !TeX root = ../presentazione.tex
% !TEX encoding = UTF-8 Unicode
% !TEX program = arara
% !TEX TS-program = arara
% !TeX spellcheck = it-IT

\centering
\begin{tabular}{ccccc}
    \multicolumn{2}{c}{\textbf{Con render}} &  & \multicolumn{2}{c}{\textbf{Senza render}} \\
     &  &  &  &  \\ \cline{1-2} \cline{4-5}
    \multicolumn{2}{|c|}{\textbf{Nuova GUI}} & \multicolumn{1}{c|}{} & \multicolumn{2}{c|}{\textbf{Nuova GUI}} \\ \cline{1-2} \cline{4-5}
    \multicolumn{1}{|c|}{Nodi + Link:} & \multicolumn{1}{c|}{$1,3153ms$} & \multicolumn{1}{c|}{} & \multicolumn{1}{c|}{Nodi + Link:} & \multicolumn{1}{c|}{1,2648ms} \\ \cline{1-2} \cline{4-5}
    \multicolumn{1}{|c|}{Solo Nodi:} & \multicolumn{1}{c|}{$0,3041ms$} & \multicolumn{1}{c|}{} & \multicolumn{1}{c|}{Solo Nodi:} & \multicolumn{1}{c|}{$0,2855ms$} \\ \cline{1-2} \cline{4-5}
     &  &  &  &  \\ \cline{1-2} \cline{4-5}
    \multicolumn{2}{|c|}{\textbf{Vecchia GUI}} & \multicolumn{1}{c|}{} & \multicolumn{2}{c|}{\textbf{Vecchia GUI}} \\ \cline{1-2} \cline{4-5}
    \multicolumn{1}{|c|}{Nodi + Link:} & \multicolumn{1}{c|}{$1,1059ms$} & \multicolumn{1}{c|}{} & \multicolumn{1}{c|}{Nodi + Link:} & \multicolumn{1}{c|}{$0,6584ms$} \\ \cline{1-2} \cline{4-5}
    \multicolumn{1}{|c|}{Solo Nodi:} & \multicolumn{1}{c|}{$0,6841ms$} & \multicolumn{1}{c|}{} & \multicolumn{1}{c|}{Solo Nodi:} & \multicolumn{1}{c|}{$0,6696ms$} \\ \cline{1-2} \cline{4-5}
\end{tabular}

    \end{frame}
\end{document}
