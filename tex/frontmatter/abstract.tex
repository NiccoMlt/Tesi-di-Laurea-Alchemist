%% Direttive TeXworks:
% !TeX root = ../../maltoni_niccolo_tesi.tex
% !TEX encoding = UTF-8 Unicode
% !TEX program = arara
% !TEX TS-program = arara
% !TeX spellcheck = it-IT

\begin{abstract}
    Lo scopo di questa tesi è la progettazione e la successiva implementazione di un'interfaccia grafica 2D per il simulatore Alchemist.
    La nuova interfaccia permette di interagire con la simulazione a tempo di esecuzione e di vedere chiaramente rappresentate informazioni su di essa.

    In particolare, è supportata una struttura modulare di effetti che rende facilmente osservabili determinate entità del sistema ed eventuali loro proprietà:
    rispetto alla classe di effetti dell'interfaccia classica, la nuova tipologia di effetti non è più una funzione dal singolo nodo alla rappresentazione grafica, bensì fa riferimento all'intero ambiente, permettendo di costruire rendering complessi facilmente esportabili tramite file JSON.

    Si è scelto di mantenere un'interfaccia il più possibile \engEmph{user-friendly}, mantenendo un design più simile ai simulatori a scopo videoludico per favorire l'utilizzo da parte di utenti inesperti.
    Lo stile estetico al quale si è deciso di allinearsi è il Material Design di Google e la libreria grafica utilizzata per l'implementazione è stata JavaFX.
\end{abstract}
