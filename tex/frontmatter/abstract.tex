%% Direttive TeXworks:
% !TeX root = ../maltoni_niccolo_tesi.tex
% !TEX encoding = UTF-8 Unicode
% !TEX program = arara
% !TEX TS-program = arara
% !TeX spellcheck = it-IT

%% Direttive Arara:
% arara: pdflatex: { shell: yes, synctex: yes, action: batchmode, options: "-halt-on-error -file-line-error-style" }
% arara: frontespizio
% arara: bibtex
% arara: pdflatex: { shell: yes, synctex: yes, action: batchmode, options: "-halt-on-error -file-line-error-style" }
% arara: pdflatex: { shell: yes, synctex: yes, action: nonstopmode, options: "-halt-on-error -file-line-error-style" }

\begin{abstract}
    \addcontentsline{toc}{chapter}{\abstractname}
    Lo scopo di questa tesi verte intorno allo studio del simulatore Alchemist e al fine di progettare un’interfaccia 2D potenziata per l'ambiente grafico relativo alla simulazione. La nuova interfaccia permette di interagire con la simulazione a tempo di esecuzione e di vedere chiaramente rappresentate informazioni su di essa; in particolare, è supportata una struttura modulare di effetti che per rendere ancora più facilmente osservabili determinate entità del sistema ed eventuali loro proprietà. Si è scelto di mantenere un'interfaccia il più possibile \engEmph{user-friendly}, mantenendo un design più simile ai simulatori a scopo videoludico per favorire l'utilizzo da parte di utenti inesperti.

    % TODO analizza risultati

    \medskip

    La seguente trattazione è strutturata su tre capitoli: nel capitolo~\vref{ch:intro} viene introdotto il contesto nel quale il lavoro descritto nella tesi ha preso parte, introducendo il simulatore Alchemist, la sua interfaccia grafica classica e il framework JavaFX; nel capitolo~\vref{ch:contributo} si espone l'intero contributo fornito al progetto, analizzando singolarmente le fasi di analisi dei requisiti, design e progettazione e in ultimo implementazione della nuova interfaccia; infine, il capitolo~\vref{ch:conclusioni} analizza i risultati ottenuti, interpretandoli anche in ottica di miglioramenti futuri.
\end{abstract}
