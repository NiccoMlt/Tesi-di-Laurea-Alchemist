%% Direttive TeXworks:
% !TeX root = ../../maltoni_niccolo_tesi.tex
% !TEX encoding = UTF-8 Unicode
% !TEX program = arara
% !TEX TS-program = arara
% !TeX spellcheck = it-IT

%% Direttive Arara:
% arara: pdflatex: { shell: yes, synctex: yes, action: batchmode, options: "-halt-on-error -file-line-error-style" }
% arara: frontespizio
% arara: biber
% arara: pdflatex: { shell: yes, synctex: yes, action: batchmode, options: "-halt-on-error -file-line-error-style" }
% arara: pdflatex: { shell: yes, synctex: yes, action: nonstopmode, options: "-halt-on-error -file-line-error-style" }

\chapter{Introduzione}\label{ch:intro}

\medskip

La seguente trattazione è strutturata su tre capitoli: nel \Cref{ch:background} viene introdotto il contesto nel quale il lavoro descritto nella tesi ha preso parte, introducendo il simulatore Alchemist, la sua interfaccia grafica classica e il framework JavaFX; nel \Cref{ch:contributo} si espone l'intero contributo fornito al progetto, analizzando singolarmente le fasi di analisi dei requisiti, design e progettazione e in ultimo implementazione della nuova interfaccia; infine, il \Cref{ch:conclusioni} analizza i risultati ottenuti, interpretandoli anche in ottica di miglioramenti futuri.
