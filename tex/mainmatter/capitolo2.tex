%% Direttive TeXworks:
% !TeX root = ../maltoni_niccolo_tesi.tex
% !TEX encoding = UTF-8 Unicode
% !TEX program = arara
% !TEX TS-program = arara
% !TeX spellcheck = it-IT

%% Direttive Arara:
% arara: pdflatex: { shell: yes, synctex: yes, action: batchmode, options: "-halt-on-error -file-line-error-style" }
% arara: frontespizio
% arara: biber
% arara: pdflatex: { shell: yes, synctex: yes, action: batchmode, options: "-halt-on-error -file-line-error-style" }
% arara: pdflatex: { shell: yes, synctex: yes, action: nonstopmode, options: "-halt-on-error -file-line-error-style" }

\chapter{Contributo}\label{ch:contributo}
  \section{Analisi dei requisiti}\label{sec:analisi}
    \subsection{Requisiti funzionali}\label{sub:funzionali}
    \subsection{Requisiti non funzionali}\label{sub:nonFunzionali}
  \section{Fonti d'ispirazione}\label{sec:ispirazione}
    \subsection{Simulatori a scopo videoludico}\label{sub:videogame}
      \subsubsection{Universe Sandbox}\label{subsub:us1}
      \subsubsection{Universe Sandbox 2}\label{subsub:us2}
      \subsubsection{SimCity}\label{subsub:simcity}
    \subsection{Material Design}\label{sub:material}
  \section{Design dell'interfaccia}\label{sec:design}
  \section{Progettazione}\label{sec:progettazione}
      \subsection{La barra inferiore}\label{sub:barra}
      \subsection{La struttura a drawer}\label{sub:drawer}
      \subsection{L'architettura degli effetti}\label{sub:effetti}
      % \subsubsection{I gruppi di effetti e l'interfaccia \texttt{EffectGroup}}\label{subsub:effectGroup}
      % \subsubsection{I singoli effetti e l'interfaccia \texttt{EffectFX}}\label{subsub:effectFX}
      % \subsubsection{Caricamento, salvataggio e modifica di gruppi di effetti}\label{subsub:serializzazione}
  \section{Dettagli implementativi}\label{sec:dettagli}
    % \subsection{Librerie utilizzate}\label{sub:lib}
    % \subsection{Gestione della concorrenza}\label{sub:concorrenza}
