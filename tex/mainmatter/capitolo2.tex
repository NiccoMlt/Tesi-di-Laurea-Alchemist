%% Direttive TeXworks:
% !TeX root = ../maltoni_niccolo_tesi.tex
% !TEX encoding = UTF-8 Unicode
% !TEX program = arara
% !TEX TS-program = arara
% !TeX spellcheck = it-IT

%% Direttive Arara:
% arara: pdflatex: { shell: yes, synctex: yes, action: batchmode, options: "-halt-on-error -file-line-error-style" }
% arara: frontespizio
% arara: bibtex
% arara: pdflatex: { shell: yes, synctex: yes, action: batchmode, options: "-halt-on-error -file-line-error-style" }
% arara: pdflatex: { shell: yes, synctex: yes, action: nonstopmode, options: "-halt-on-error -file-line-error-style" }

\chapter{Contributo}\label{ch:contributo}
\section{Analisi dei requisiti}\label{sec:analisi}
\subsection{Requisiti funzionali}\label{sub:funzionali}
\subsection{Requisiti non funzionali}\label{sub:nonFunzionali}
\section{Fonti d'ispirazione}\label{sec:ispirazione}
\subsection{Simulatori a scopo videoludico}\label{sub:videogame}
\subsubsection{Universe Sandbox}\label{subsub:us1}
\subsubsection{Universe Sandbox 2}{Universe Sandbox\textsuperscript{2}}\label{subsub:us2}
\subsubsection{SimCity}\label{subsub:simcity}
\subsection{Material Design}\label{sub:material}
\section{Design dell'interfaccia}\label{sec:design}
\section{Progettazione}\label{sec:progettazione}
\subsubsection{I gruppi di effetti e l'interfaccia \texttt{EffectGroup}}\label{subsub:effectGroup}
\subsubsection{I singoli effetti e l'interfaccia \texttt{EffectFX}}\label{subsub:effectFX}
\subsubsection{}
\subsubsection{Differenze di serializzazione}\label{subsub:serializzazione}
\subsection{Rappresentazione grafica e concorrenza}\label{sub:rappresentazione}
