%% Direttive TeXworks:
% !TeX root = ../maltoni_niccolo_tesi.tex
% !TEX encoding = UTF-8 Unicode
% !TEX program = arara
% !TEX TS-program = arara
% !TeX spellcheck = it-IT

%% Direttive Arara:
% arara: pdflatex: { shell: yes, synctex: yes, action: batchmode, options: "-halt-on-error -file-line-error-style" }
% arara: frontespizio
% arara: bibtex
% arara: pdflatex: { shell: yes, synctex: yes, action: batchmode, options: "-halt-on-error -file-line-error-style" }
% arara: pdflatex: { shell: yes, synctex: yes, action: nonstopmode, options: "-halt-on-error -file-line-error-style" }

\chapter{Contributo}\label{ch:contributo}
\section{Analisi della struttura precedente}\label{sec:analisi}
\subsection{User experience}\label{sub:prevUx}
\subsection{Swing}\label{sub:swing}
\subsection{Gli effetti e l'interfaccia \texttt{Effect}}\label{sub:effect}
\section{Nuova architettura}\label{sec:architettura}
\subsection{I vantaggi di JavaFX su Swing}\label{sub:swingVSjfx}
\subsection{Interfaccia e UX}\label{sub:gui}
\subsection{La nuova architettura degli effetti}\label{sub:nuoviEffetti}
\subsubsection{I gruppi di effetti e l'interfaccia \texttt{EffectGroup}}\label{subsub:effectGroup}
\subsubsection{I singoli effetti e l'interfaccia \texttt{EffectFX}}\label{subsub:effectFX}
\subsubsection{Differenze di serializzazione}\label{subsub:serializzazione}
\subsection{Rappresentazione grafica e concorrenza}\label{sub:rappresentazione}
