%% Direttive TeXworks:
% !TeX root = ../maltoni_niccolo_tesi.tex
% !TEX encoding = UTF-8 Unicode
% !TEX program = arara
% !TEX TS-program = arara
% !TeX spellcheck = it-IT

%% Direttive Arara:
% arara: pdflatex: { shell: yes, synctex: yes, action: batchmode, options: "-halt-on-error -file-line-error-style" }
% arara: frontespizio
% arara: biber
% arara: pdflatex: { shell: yes, synctex: yes, action: batchmode, options: "-halt-on-error -file-line-error-style" }
% arara: pdflatex: { shell: yes, synctex: yes, action: nonstopmode, options: "-halt-on-error -file-line-error-style" }

\chapter{Contributo}\label{ch:contributo}
    In questo capitolo verrà analizzato il contributo fornito al progetto, elencando i requisiti necessari e analizzando il processo di soddisfazione degli stessi.

    L'obiettivo principale è quello di integrare una nuova interfaccia per la simulazione, al fine di semplificare l’adozione del simulatore da parte di utenti inesperti.

    \section{Analisi dei requisiti}\label{sec:analisi}
        Lo studio del lavoro illustrato in questa tesi ha inizio con l'analisi dei requisiti dell’interfaccia utente, ossia cosa l'applicazione deve mostrare a schermo.

        Questa sezione si occuperà di enunciare i requisiti funzionali e non funzionali individuati.

        \subsection{Requisiti funzionali}\label{sub:funzionali}
            I requisiti funzionali descrivono il comportamento che il sistema deve avere

        \subsection{Requisiti non funzionali}\label{sub:nonFunzionali}
    \section{Fonti d'ispirazione}\label{sec:ispirazione}
        \subsection{Simulatori a scopo videoludico}\label{sub:videogame}
            \subsubsection{Universe Sandbox}\label{subsub:us1}
            \subsubsection{Universe Sandbox 2}\label{subsub:us2}
            \subsubsection{SimCity}\label{subsub:simcity}
        \subsection{Material Design}\label{sub:material}
    \section{Design dell'interfaccia}\label{sec:design}
    \section{Progettazione}\label{sec:progettazione}
        \subsection{La barra inferiore}\label{sub:barra}
        \subsection{La struttura a drawer}\label{sub:drawer}
        \subsection{L'architettura degli effetti}\label{sub:effetti}
            % \subsubsection{I gruppi di effetti e l'interfaccia \texttt{EffectGroup}}\label{subsub:effectGroup}
            % \subsubsection{I singoli effetti e l'interfaccia \texttt{EffectFX}}\label{subsub:effectFX}
            % \subsubsection{Caricamento, salvataggio e modifica di gruppi di effetti}\label{subsub:serializzazione}
    \section{Dettagli implementativi}\label{sec:dettagli}
        % \subsection{Librerie utilizzate}\label{sub:lib}
        % \subsection{Gestione della concorrenza}\label{sub:concorrenza}
