%% Direttive TeXworks:
% !TeX root = ../maltoni_niccolo_tesi.tex
% !TEX encoding = UTF-8 Unicode
% !TEX program = arara
% !TEX TS-program = arara
% !TeX spellcheck = it-IT

%% Direttive Arara:
% arara: pdflatex: { shell: yes, synctex: yes, action: batchmode, options: "-halt-on-error -file-line-error-style" }
% arara: frontespizio
% arara: bibtex
% arara: pdflatex: { shell: yes, synctex: yes, action: batchmode, options: "-halt-on-error -file-line-error-style" }
% arara: pdflatex: { shell: yes, synctex: yes, action: nonstopmode, options: "-halt-on-error -file-line-error-style" }

\chapter{Introduzione}\label{ch:intro}
  \section{Alchemist}\label{sec:alchemist}
    \subsection{Introduzione ad Alchemist}\label{sub:introAlchemist}
    \subsection{Modello computazionale di Alchemist}\label{sub:modelloComputazionale}
    \subsection{Interfaccia utente classica}\label{sub:prevGui}
      \subsubsection{Esperienza utente}\label{subsub:prevUx}
      \subsubsection{Swing}\label{subsub:swing}
      \subsubsection{Gli effetti e l'interfaccia \texttt{Effect}}\label{subsub:effect}
  \section{JavaFX}\label{sec:jfx}
    \subsection{Introduzione a JavaFX}\label{sub:jfxIntro}
    \subsection{Il framework JavaFX}\label{sub:jfxFramework}
    \subsection{Struttura di una Applicazione JavaFX}\label{sub:jfxStruttura}
    \subsection{Vantaggi di JavaFX su Swing}\label{sub:jfxVantaggi}
  \section{Interfaccia JavaFX per Alchemist: motivazioni}\label{sec:motivi}
