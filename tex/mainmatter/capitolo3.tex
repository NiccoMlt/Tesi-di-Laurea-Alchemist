%% Direttive TeXworks:
% !TeX root = ../maltoni_niccolo_tesi.tex
% !TEX encoding = UTF-8 Unicode
% !TEX program = arara
% !TEX TS-program = arara
% !TeX spellcheck = it-IT

%% Direttive Arara:
% arara: pdflatex: { shell: yes, synctex: yes, action: batchmode, options: "-halt-on-error -file-line-error-style" }
% arara: frontespizio
% arara: biber
% arara: pdflatex: { shell: yes, synctex: yes, action: batchmode, options: "-halt-on-error -file-line-error-style" }
% arara: pdflatex: { shell: yes, synctex: yes, action: nonstopmode, options: "-halt-on-error -file-line-error-style" }

\chapter{Conclusioni}\label{ch:conclusioni}
    \section{Risultati}\label{sec:risultati}
        L'obiettivo di questa tesi era quello di realizzare un'interfaccia grafica per l'ambiente di simulazione di Alchemist che sostituisse la precedente e si andasse ad integrare con i recenti contributi dati ad altre sezioni della GUI del software, andando a fornire una esperienza utente più semplice e gradevole anche per i meno esperti.

        L'interfaccia, realizzata con la libreria JavaFX, ha comportato un restyling completo a livello estetico e una reimplementazione di numerose parti del codice; i requisiti sono quasi completamente soddisfatti, fatta eccezione per l'ambiente di simulazione (\textbf{\texttt{e tutto il resto che al momento manca [ndr.]}}).
        Nonostante quest'ultima mancanza non permetta il completo abbandono dell'interfaccia precedente, il codice realizzato adempie con correttezza ai compiti per cui è stato scritto, dunque l'implementazione può considerarsi riuscita.

        \textbf{\texttt{[Continua...]}}

    \section{Lavori futuri}\label{sec:futuro}
        \textbf{\texttt{[...]}}
