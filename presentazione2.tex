%% Direttive TeXworks:
% !TeX root = ./presentazione.tex
% !TEX encoding = UTF-8 Unicode
% !TEX program = arara
% !TEX TS-program = arara
% !TeX spellcheck = it-IT

% arara: pdflatex: { synctex: yes, action: batchmode, options: "-halt-on-error -file-line-error-style" }
% arara: pdflatex: { synctex: yes, action: nonstopmode, options: "-halt-on-error -file-line-error-style" }

\documentclass[%
    % handout,      % serve per generare la versione stampabile
    % italian
]{beamer}

%% ORDINE IMPORTANTE INIZIO %%%%%%%%%%%%
\usepackage[T1]{fontenc}        % serve per impostare la codifica di output del font
\usepackage{textcomp}           % serve per fornire supporto ai Text Companion fonts
\usepackage[utf8]{inputenc}     % serve per impostare la codifica di input del font
\usepackage[
    english,        % utilizza l'inglese come lingua secondaria
    italian         % utlizza l'italiano come lingua primaria
]{babel}                        % serve per scrivere Indice, Capitolo, etc in Italiano
\usepackage{lmodern}            % carica una variante Latin Modern prodotto dal GUST
%% ORDINE IMPORTANTE FINE %%%%%%%%%%%%%%
\usepackage{xcolor}             % serve per la gestione dei colori nel testo
\usepackage{graphicx}           % serve per includere immagini e grafici
\usepackage{wrapfig}            % serve per includere figure "wrapped" nel testo
\usepackage[labelformat=empty]{caption}
\usepackage{subcaption}

\usepackage[%
    strict,             % rende tutti gli warning degli errori
    autostyle,          % imposta lo stile in base al linguaggio specificato in babel
    english=american,   % imposta lo stile per l'inglese
    italian=guillemets  % imposta lo stile per l'italiano
]{csquotes}                     % serve a impostare lo stile delle virgolette

% \usetheme{CambridgeUS}        % TODO decidi quale usare
\usetheme{Boadilla}             % serve per scegliere il layout generale dei frame
\usecolortheme{beaver}          % Per il colore va comunque bene questo

\usepackage{pgf}
% \usepackage{pgfpages}         % serve per mettere più slide in una pagina
% \pgfpagesuselayout{4 on 1}[a4paper, border shrink=5mm, landscape]

% \usepackage[htt]{hyphenat}      % Enable hyphenation of TT text
\hyphenation{                   % Permette di sillabare bene le parole
    JavaFX
    Swing
    Micro-systems
    Script
    script
    Pack-age
    pack-age
    Manage-ment
    manage-ment
    Com-munity
    com-munity
}

\setcounter{secnumdepth}{2}     % Numera fino alla sottosezione nel corpo del testo
\setcounter{tocdepth}{3}        % Numera fino alla sotto-sottosezione nell'indice

\graphicspath{{img/}}

%% Definisco un nuovo comando per enfatizzare il testo in inglese %%%%%%%%%%%
\newcommand{\engEmph}[1] {\emph{\foreignlanguage{english}#1}}

% hyperref va caricato il più tardi possibile
\usepackage{hyperxmp}           % Simile ad hyperref
\usepackage{hyperref}           % Gestisce tutte le cose ipertestuali del pdf
\hypersetup{%
    pdfpagemode={UseOutlines},
    hidelinks,          % nasconde i collegamenti (non vengono quadrettati)
    hypertexnames=false,
    linktoc=all,        % inserisce i link nell'indice
    bookmarks=true,
    bookmarksopen,
    bookmarksopenlevel=3,
    plainpages=false,
    breaklinks,
    pdfstartview={FitH},
    pdfauthor={Niccolò Maltoni},
    pdfcopyright={Copyright (C) 2017, Niccolò Maltoni},
    pdftitle={Progettazione object-oriented di un'interfaccia grafica JavaFX per il simulatore Alchemist},
    pdfcontactcountry={Italy},
    pdfcontactemail={niccolo.maltoni@studio.unibo.it},
    pdflang={it},
    pdfsubject={Lo scopo di questa tesi è la progettazione e la successiva implementazione di un'interfaccia grafica 2D per il simulatore Alchemist. La nuova interfaccia permette di interagire con la simulazione a tempo di esecuzione e di vedere chiaramente rappresentate informazioni su di essa. In particolare, è supportata una struttura modulare di effetti che rende facilmente osservabili determinate entità del sistema ed eventuali loro proprietà: rispetto alla classe di effetti dell'interfaccia classica, la nuova tipologia di effetti non è più una funzione dal singolo nodo alla rappresentazione grafica, bensì fa riferimento all'intero ambiente, permettendo di costruire rendering complessi facilmente esportabili tramite file JSON. Si è scelto di mantenere un'interfaccia il più possibile user-friendly, mantenendo un design più simile ai simulatori a scopo videoludico per favorire l'utilizzo da parte di utenti inesperti. Lo stile estetico al quale si è deciso di allinearsi è il Material Design di Google e la libreria grafica utilizzata per l'implementazione è stata JavaFX.},
    pdfkeywords={Progettazione object-oriented, Simulazione, Java, JavaFX, Interfaccia grafica},
    pdfa
}

\title[Nuova GUI Alchemist in JavaFX]{%
    Progettazione object-oriented di un'interfaccia grafica %
    JavaFX per il simulatore Alchemist
}
\subtitle{Tesi in Programmazione ad Oggetti}
\author[Niccolò~Maltoni]{Niccolò~Maltoni\\0000719734}
% \author[%
%     Niccolò~Maltoni \hspace{0.1\textwidth} \insertframenumber\,/\,\inserttotalframenumber%
% ]{%
%     Niccolò~Maltoni%
%     \\ \vspace{10pt} \small{Realtore: Prof.~Mirko~Viroli \\Correlatore: Prof.~Danilo~Pianini}%
% }
\date{12 dicembre 2017}
\institute[]{%
    Alma Mater Studiorum - Università di Bologna\\%
    Campus di Cesena%
}

%% Permette di inserire l'outline prima di ogni sezione
\AtBeginSection[]{%
    \begin{frame}<beamer>
        \frametitle{Outline}
        \tableofcontents[currentsection]
    \end{frame}
}

\begin{document}
    \begin{frame}
        \titlepage
    \end{frame}

    \section{Introduzione}\label{sec:intro}
        \subsection{Obiettivo}\label{subsec:obiettivo}
            \begin{frame}
                \frametitle{\insertsection}
                \framesubtitle{\insertsubsection}
                Lo scopo di questa tesi è la progettazione e la successiva implementazione di un'interfaccia grafica 2D per il simulatore \emph{Alchemist}.

                \medskip
                \pause

                La nuova interfaccia ha come obiettivo l'interazione con la simulazione a tempo di esecuzione e la rappresentazione di informazioni su di essa attraverso una GUI il più possibile \engEmph{user-friendly}.
            \end{frame}

        \subsection{Alchemist}\label{subsec:alchemist}
            \begin{frame}
                \frametitle{\insertsection}
                \framesubtitle{\insertsubsection}

                Alchemist\only<1->{\footnote{\url{http://alchemistsimulator.github.io}}} è un meta-simulatore estendibile completamente open-source che esegue su Java Virtual Machine (JVM), nato all’interno dell’Università di Bologna e reperibile su GitHub\only<1->{\footnote{\url{https://github.com/AlchemistSimulator/Alchemist}}}.
            \end{frame}

    \section{Analisi}\label{sec:analisi}
        \subsection{L'interfaccia classica di Alchemist}\label{subsec:oldgui}
            % \begin{frame}
            %     \frametitle{\insertsection}
            %     L'architettura di Alchemist è progettata con paradigma \engEmph{Model-View-Controller} (MVC), di conseguenza la suddivisione tra componente grafica (\engEmph{View}) e il blocco ``logico'' composto da \engEmph{Model} e \engEmph{Controller} è netta.
            % \end{frame}

            \begin{frame}
                \frametitle{\insertsection}
                \framesubtitle{\insertsubsection}
                \centering
                \frame{\includegraphics[scale=.27]{old/window_stopped_circle}}
            \end{frame}

            \begin{frame}
                \frametitle{\insertsection}
                \framesubtitle{\insertsubsection}
                L'interfaccia utente classica di Alchemist ha diverse criticità:

                \begin{itemize}[<+(1)->]
                  \item
                      Il sistema di controllo non è intuitivo;
                  \item
                      L'aspetto estetico è datato e non aderisce ad alcun design grafico in particolare;
                  \item
                      Le capacità di rappresentazione, rappresentate dagli effetti, sono legate strettamente ai nodi e limitano la libertà di rappresentazione.
                \end{itemize}
            \end{frame}

        \subsection{Requisiti \& Casi d'uso}\label{subsec:requisiti}
            \begin{frame}
                \frametitle{\insertsection}
                \framesubtitle{\insertsubsection}
                \centering
                \includegraphics[scale=0.40]{uml/useCaseBeamer}
            \end{frame}

    \section{Design \& Mockup}\label{sec:mockup}
        \subsection{Stato dell'arte e fonti d'ispirazione}\label{subsec:sota}
            \begin{frame}
                \frametitle{\insertsection}
                \framesubtitle{\insertsubsection}

                \begin{itemize}[<+->]
                    \item
                        Poiché l'obiettivo era semplificare quanto più possibile l'interazione, non sono state considerate le interfacce dei simulatori a scopo scientifico costituenti lo stato dell'arte \ldots
                    \item
                        \ldots bensì si è preferito ispirarsi a famosi simulatori videoludici, quali ad esempio Universe Sandbox.
                    \item
                        Lo stile grafico è basato sul Material Design di Google.
                \end{itemize}
            \end{frame}

        \subsection{Mockup}\label{subsec:mockup}
            \begin{frame}
                \frametitle{\insertsection}
                \framesubtitle{\insertsubsection}
                \centering
                \includegraphics[scale=0.415]{mock/withNodes/window}
            \end{frame}

            \begin{frame}
                \begin{figure}[htbp]
                    \captionsetup[subfigure]{labelformat=empty}
                    \centering%
                    \begin{subfigure}{.2\textwidth}
                        \includegraphics[scale=0.4]{mock/crop/groups}
                        \caption{Gruppi di effetti dello \engEmph{stack}}
                        \label{fig:mock:groups}
                    \end{subfigure}
                    \qquad{\LARGE$\Rightarrow$}\qquad
                    \pause
                    \begin{subfigure}{.2\textwidth}
                        \includegraphics[scale=0.4]{mock/crop/effects}
                        \caption{Effetti di un gruppo}
                        \label{fig:mock:effects}
                    \end{subfigure}
                    \qquad{\LARGE$\Rightarrow$}\qquad
                    \pause
                    \begin{subfigure}{.2\textwidth}
                        \includegraphics[scale=0.4]{mock/crop/properties}
                        \caption{Proprietà di un effetto}
                        \label{fig:mock:properties}
                    \end{subfigure}
                \end{figure}
            \end{frame}

    \section{L'interfaccia implementata}\label{sec:newgui}
        \subsection{Caratteristiche e miglioramenti}\label{subsec:feature}
            \begin{frame}
                \frametitle{\insertsection}
                \framesubtitle{\insertsubsection}
                \centering \huge TODO immagine% TODO inserisci screenshot
            \end{frame}

            \begin{frame}
                \frametitle{\insertsection}
                \framesubtitle{\insertsubsection}
                \begin{columns}[onlytextwidth]
                    \begin{column}{.4\textwidth}
                    \begin{itemize}
                        \item<1->
                            è stata riprogettata l'architettura degli effetti;
                        \item<2->
                            è stato introdotto l'uso di proprietà specifiche per gli effetti;
                        \item<3->
                            sono stati implementati diversi monitor per la rappresentazione grafica:
                            \begin{itemize}
                              \item<4->
                                  dello stato;
                              \item<5->
                                  dell'avanzamento;
                              \item<6->
                                  degli effetti.
                            \end{itemize}
                    \end{itemize}
                    \end{column}
                    \begin{column}{.6\textwidth}
                        \centering
                        \only<1>{\includegraphics[scale=0.35]{uml/EffectFXUMLsimple}}
                        \only<2>{Due}
                        \only<3>{Tre}
                        \only<4>{Quattro}
                        \only<5>{Cinque}
                        \only<6>{Sei}
                    \end{column}
                \end{columns}
            \end{frame}

            % TODO Caratteristiche a testo ?
            % \begin{frame}
            %     \frametitle{\insertsection}
            %     \framesubtitle{\insertsubsection}
            % \end{frame}

            % TODO UML ?
            % \begin{frame}
            %     \frametitle{\insertsection}
            %     \framesubtitle{\insertsubsection}
            % \end{frame}

        \subsection{Coverage}\label{subsec:coverage}
            \begin{frame}
                \frametitle{\insertsection}
                \framesubtitle{\insertsubsection}
                \centering
                \frame{\includegraphics[scale=0.43]{fig/filtered_coverage}}
            \end{frame}

        \subsection{Prestazioni}\label{subsec:benchmark}
            \begin{frame}
                \frametitle{\insertsection}
                \framesubtitle{\insertsubsection}
                \centering
                %% Direttive TeXworks:
% !TeX root = ../presentazione.tex
% !TEX encoding = UTF-8 Unicode
% !TEX program = arara
% !TEX TS-program = arara
% !TeX spellcheck = it-IT

\centering
\begin{tabular}{ccccc}
    \multicolumn{2}{c}{\textbf{Con render}} &  & \multicolumn{2}{c}{\textbf{Senza render}} \\
     &  &  &  &  \\ \cline{1-2} \cline{4-5}
    \multicolumn{2}{|c|}{\textbf{Nuova GUI}} & \multicolumn{1}{c|}{} & \multicolumn{2}{c|}{\textbf{Nuova GUI}} \\ \cline{1-2} \cline{4-5}
    \multicolumn{1}{|c|}{Nodi + Link:} & \multicolumn{1}{c|}{$1,3153ms$} & \multicolumn{1}{c|}{} & \multicolumn{1}{c|}{Nodi + Link:} & \multicolumn{1}{c|}{1,2648ms} \\ \cline{1-2} \cline{4-5}
    \multicolumn{1}{|c|}{Solo Nodi:} & \multicolumn{1}{c|}{$0,3041ms$} & \multicolumn{1}{c|}{} & \multicolumn{1}{c|}{Solo Nodi:} & \multicolumn{1}{c|}{$0,2855ms$} \\ \cline{1-2} \cline{4-5}
     &  &  &  &  \\ \cline{1-2} \cline{4-5}
    \multicolumn{2}{|c|}{\textbf{Vecchia GUI}} & \multicolumn{1}{c|}{} & \multicolumn{2}{c|}{\textbf{Vecchia GUI}} \\ \cline{1-2} \cline{4-5}
    \multicolumn{1}{|c|}{Nodi + Link:} & \multicolumn{1}{c|}{$1,1059ms$} & \multicolumn{1}{c|}{} & \multicolumn{1}{c|}{Nodi + Link:} & \multicolumn{1}{c|}{$0,6584ms$} \\ \cline{1-2} \cline{4-5}
    \multicolumn{1}{|c|}{Solo Nodi:} & \multicolumn{1}{c|}{$0,6841ms$} & \multicolumn{1}{c|}{} & \multicolumn{1}{c|}{Solo Nodi:} & \multicolumn{1}{c|}{$0,6696ms$} \\ \cline{1-2} \cline{4-5}
\end{tabular}

            \end{frame}
\end{document}
